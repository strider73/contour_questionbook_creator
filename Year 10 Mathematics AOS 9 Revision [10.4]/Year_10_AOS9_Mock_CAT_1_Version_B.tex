\documentclass[12pt,a4paper]{article}
\usepackage[utf8]{inputenc}
\usepackage[margin=2cm]{geometry}
\usepackage{amsmath}
\usepackage{amssymb}
\usepackage{graphicx}
\usepackage{enumitem}
\usepackage{fancyhdr}

\pagestyle{fancy}
\fancyhf{}
\rhead{Version B}
\lhead{Year 10 Mathematics AOS 9 Revision [10.4]}
\cfoot{\thepage}

\title{\textbf{Year 10 Mathematics}\\AOS 9 Revision [10.4] Mock CAT 1\\Version B}
\author{}
\date{}

\begin{document}

\maketitle

\begin{center}
\textbf{Instructions:} Answer all questions. Show all working.\\
\textbf{Total Marks:} 50\\
\textbf{Time:} 60 minutes
\end{center}

\vspace{1cm}

\section*{Section A: Short Answer Questions (36 Marks)}

\subsection*{Question 1 [1 mark]}

Write the intersection $(-\infty, 5] \cap [0, \infty)$ in interval notation.

\begin{center}
\includegraphics[width=0.9\textwidth]{new_images/q1_number_line.png}
\end{center}

\vspace{1cm}

\subsection*{Question 2 [1 mark]}

Factorise completely: $16x^2 - 49$.

\vspace{2cm}

\subsection*{Question 3 [1 mark]}

Does the relation $\{(3, 1), (-2, 4), (3, -1)\}$ define a function? Give a reason.

\begin{center}
\includegraphics[width=0.5\textwidth]{new_images/q3_function_graph.png}
\end{center}

\vspace{1cm}

\subsection*{Question 4 [1 mark]}

Expand: $(x - 2)(x + 7)$.

\vspace{2cm}

\subsection*{Question 5 [1 mark]}

For $y = (x + 2)^2 - 5$, find the y-intercept.

\vspace{2cm}

\newpage

\subsection*{Question 6 [2 marks]}

Use long division to find the quotient and remainder when $3x^4 + 2x^3 - 4x + 1$ is divided by $x^2 - 1$.

\vspace{4cm}

\subsection*{Question 7 [2 marks]}

Solve $x^3 + 2x^2 - 4x - 8 = 0$.

\vspace{4cm}

\subsection*{Question 8 [2 marks]}

Sketch $y = (x + 2)^2(x - 2)$. Label all intercepts.

\begin{center}
\includegraphics[width=0.6\textwidth]{new_images/q8_cubic_graph.png}
\end{center}

\vspace{2cm}

\newpage

\subsection*{Question 9 [2 marks]}

Find the centre and radius of $x^2 + y^2 - 6x + 8y - 11 = 0$.

\vspace{4cm}

\subsection*{Question 10 [2 marks]}

Let $f(x) = 2x^3 + 5x^2 + x - 2$. Using the Remainder Theorem, find the remainder on division by $x + 1$. Hence, decide if $x + 1$ is a factor.

\vspace{4cm}

\subsection*{Question 11 [2 marks]}

Sketch $y = (x - 1)^4 - 2$. Label the turning point and any intercepts.

\begin{center}
\includegraphics[width=0.6\textwidth]{new_images/q11_quartic_graph.png}
\end{center}

\vspace{2cm}

\newpage

\subsection*{Question 12 [2 marks]}

Find $k$ so that $(x - 2)$ is a factor of $x^3 - kx^2 + x + 6$. Hence, factor the polynomial completely.

\vspace{4cm}

\subsection*{Question 13 [2 marks]}

Sketch the graph $y = \frac{-3}{x+2} + 1$. Label the asymptotes.

\begin{center}
\includegraphics[width=0.6\textwidth]{new_images/q13_rational_graph.png}
\end{center}

\vspace{2cm}

\newpage

\subsection*{Question 14 [4 marks]}

From the platform edge at $x = -2$ m to $x = 7$ m along the deck, the slide height (m) is $h(x) = \sqrt{x + 2} - 3$.

\textbf{a.} State the domain and range over this section. \textbf{(2 marks)}

\vspace{3cm}

\textbf{b.} How far from the edge ($x = -2$) does the slide first reach $-1$ m high? \textbf{(2 marks)}

\vspace{3cm}

\subsection*{Question 15 [4 marks]}

A circular spray pattern is $(x + 3)^2 + (y - 2)^2 = 25$ (metres).

\textbf{a.} State the centre and radius. \textbf{(1 mark)}

\vspace{2cm}

\textbf{b.} Is $(1, 5)$ on the boundary? Justify. \textbf{(1 mark)}

\vspace{2cm}

\textbf{c.} The sprinkler is moved 4 m right and 1 m down. Write the new equation. \textbf{(2 marks)}

\vspace{2cm}

\newpage

\subsection*{Question 16 [3 marks]}

A stage light intensity curve is $y = -\frac{1}{2}(x - 3)^4 + 2$.

\textbf{a.} Describe the transformations from $y = x^4$. \textbf{(2 marks)}

\vspace{3cm}

\textbf{b.} Identify the turning point and whether the curve opens up or down; justify from your description. \textbf{(1 mark)}

\vspace{2cm}

\subsection*{Question 17 [4 marks]}

A tunnel cross-section is $x^2 + y^2 = 25$ (metres), ground is $y = 0$. A truck travels along the centreline.

\textbf{a.} Determine the maximum truck width that fits at the height $y = 4$ m. \textbf{(2 marks)}

\vspace{3cm}

\textbf{b.} If the truck is 6 m wide, what is the maximum height it can have at the centreline? Give an exact value. \textbf{(2 marks)}

\vspace{3cm}

\newpage

\section*{Section B: Extended Response Questions (14 Marks)}

\subsection*{Question 18 [14 marks]}

The city is finalising Harmony Plaza, a plaza drawn on the Cartesian plane (units in metres).

The outer rim of a round seating area is a circle whose diameter has endpoints $A(-2, 3)$ and $B(4, 7)$.

\textbf{a.} Find the equation of the circle. \textbf{(3 marks)}

\vspace{4cm}

A family of straight paths is planned with equations $L_k: y = -x + k$.

\textbf{b.} Determine the value(s) of $k$ for which $L_k$ is tangent to the circle. \textbf{(3 marks)}

\begin{center}
\includegraphics[width=0.6\textwidth]{new_images/q18b_circle_tangent.png}
\end{center}

\vspace{3cm}

\textbf{c.} Hence, determine the equation of the tangent line(s). \textbf{(1 mark)}

\vspace{2cm}

\newpage

\textbf{d.} \textbf{Tech-Active.} Sketch the graph of the circle and a line $L_k$ for the $k$ found in \textbf{part b}. Label the point of intersection, correct to two decimal places. \textbf{(3 marks)}

\begin{center}
\includegraphics[width=0.7\textwidth]{new_images/q18d_coordinate_grid.png}
\end{center}

\vspace{2cm}

A raised planter's front edge is modelled by the cubic:
$$G(x) = x^3 - 5x^2 + 2x + 8$$

\textbf{e.}

\hspace{1cm}\textbf{i.} Use the Factor Theorem to show that $(x - 2)$ is a factor of $G(x)$. \textbf{(1 mark)}

\vspace{3cm}

\hspace{1cm}\textbf{ii.} Perform polynomial division to factorise $G(x)$ completely, and use the Null Factor Law to find all x-intercepts. \textbf{(3 marks)}

\vspace{4cm}

\end{document}
