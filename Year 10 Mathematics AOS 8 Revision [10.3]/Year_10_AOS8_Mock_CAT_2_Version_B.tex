\documentclass[12pt,a4paper]{article}
\usepackage[margin=2cm]{geometry}
\usepackage{amsmath}
\usepackage{graphicx}
\usepackage{enumitem}
\usepackage{fancyhdr}
\usepackage{titlesec}

% Header and footer
\pagestyle{fancy}
\fancyhf{}
\fancyhead[L]{\textbf{Year 10 Mathematics}}
\fancyhead[R]{\textbf{AOS 8 Mock CAT 2 - Version B}}
\fancyfoot[C]{\thepage}
\renewcommand{\headrulewidth}{0.4pt}
\renewcommand{\footrulewidth}{0.4pt}

% Section formatting
\titleformat{\section}
  {\normalfont\large\bfseries}{\thesection}{1em}{}
\titleformat{\subsection}
  {\normalfont\normalsize\bfseries}{\thesubsection}{1em}{}

% Question environment
\newcounter{question}
\newenvironment{question}[1]
  {\refstepcounter{question}\subsection*{Question \thequestion\quad[#1]}}
  {\vspace{1em}\hrule\vspace{1em}}

\setlength{\parindent}{0pt}
\setlength{\parskip}{0.5em}

\begin{document}

% Title section
\begin{center}
{\Large\textbf{Year 10 Mathematics}}\\[0.3em]
{\large\textbf{AOS 8 Revision [10.3] Mock CAT 2}}\\[0.3em]
{\large\textbf{Version B}}\\[1em]
\end{center}

\noindent\textbf{Instructions:} Answer all questions. Show all working.\\
\textbf{Total Marks:} 50\\
\textbf{Time:} 60 minutes writing

\vspace{1em}
\hrule
\vspace{1em}

\section*{Section A: Short Answer Questions (34 Marks)}

\hrule
\vspace{1em}

% Question 1
\begin{question}{1 mark}
Expand and simplify: $(x + 7)(x - 2)$.
\vspace{3cm}
\end{question}

% Question 2
\begin{question}{1 mark}
Find the degree of $p(x) = x^2(2x^4 - 5x + 3)$, where the constant term is an integer.
\vspace{3cm}
\end{question}

% Question 3
\begin{question}{1 mark}
State the x-intercepts of $y = x(x + 3)^2$.
\vspace{3cm}
\end{question}

% Question 4
\begin{question}{1 mark}
What is the remainder when $x^5 - 8$ is divided by $(x + 2)$?
\vspace{3cm}
\end{question}

% Question 5
\begin{question}{2 marks}
Sketch the graph of $y = (x + 3)(x - 1)(x - 4)$ on the axes below, clearly labelling all x-intercepts and the y-intercept.

\vspace{0.5cm}
\begin{center}
\includegraphics[width=0.7\textwidth]{new_images_CAT2/q5_grid.png}
\end{center}
\vspace{0.5cm}
\end{question}

% Question 6
\begin{question}{3 marks}
Solve the following polynomial equations:

\textbf{a.} $x(x + 4)(x - 7) = 0$ \hfill (1 mark)
\vspace{2cm}

\textbf{b.} $x^4 - 13x^2 + 36 = 0$ \hfill (2 marks)
\vspace{3cm}
\end{question}

% Question 7
\begin{question}{3 marks}
Given $(x + 2)$ is a factor of $P(x) = x^3 - 2x^2 - 13x - 10$, fully factorise $P(x)$.
\vspace{5cm}
\end{question}

% Question 8
\begin{question}{3 marks}
Given $P(x) = 3x^3 + 5x^2 - 26x + 8$:

\textbf{a.} Verify that $x = 2$ is a root of the equation $P(x) = 0$. \hfill (1 mark)
\vspace{2cm}

\textbf{b.} Determine the remainder when $P(x)$ is divided by $(x - 1)$. \hfill (1 mark)
\vspace{2cm}

\textbf{c.} It is known that $P(x) \div (x - a) = Q(x) + \frac{18}{x-a}$, where $Q(x)$ is a quadratic polynomial. Find $P(a)$. \hfill (1 mark)
\vspace{2cm}
\end{question}

\newpage

% Question 9
\begin{question}{3 marks}
Sketch the graph of $y = (x - 1)^4 - 16$ on the axes below, labelling all axes intercepts and turning points.

\vspace{0.5cm}
\begin{center}
\includegraphics[width=0.7\textwidth]{new_images_CAT2/q9_grid.png}
\end{center}
\vspace{0.5cm}
\end{question}

% Question 10
\begin{question}{2 marks}
Use long division to find the quotient and remainder when $3x^3 + 7x^2 - 4x + 1$ is divided by $(x + 2)$.
\vspace{5cm}
\end{question}

% Question 11
\begin{question}{2 marks}
Find the value of $k$ if $(x + 3)$ is a factor of $P(x) = x^3 + kx^2 + 7x + 12$.
\vspace{4cm}
\end{question}

% Question 12
\begin{question}{2 marks}
Expand and simplify $(x + 3)(x^2 - 4x + 5)$.
\vspace{4cm}
\end{question}

% Question 13
\begin{question}{5 marks}
The monthly revenue, $R$, in thousands of dollars, of a startup company is modelled by $R(x) = x^3 - 12x^2 + 27x$, where $x$ is the number of units sold, in hundreds.

\textbf{a.} Factorise the revenue function $R(x)$. \hfill (2 marks)
\vspace{3cm}

\textbf{b.} For what sales levels does the company earn zero revenue? \hfill (2 marks)
\vspace{3cm}

\textbf{c.} What is the revenue if 500 units are sold? \hfill (1 mark)
\vspace{2cm}
\end{question}

% Question 14
\begin{question}{5 marks}
A population model $N(x)$ (in hundreds) after $x$ months of environmental changes is modelled by $N(x) = x^3 - 3x^2 - 10x + 24$.

\textbf{a.} Using the Factor Theorem and trial-and-error, find a time $x$ at which the population returns to its initial level (where $N(x) = 24$). \hfill (2 marks)
\vspace{3cm}

\textbf{b.} Once you have this value, factorise $N(x) - 24$ completely to determine all times when the population equals 24. \hfill (3 marks)
\vspace{4cm}
\end{question}

\newpage

\section*{Section B: Extended Response Questions (16 Marks)}

\hrule
\vspace{1em}

% Question 15
\begin{question}{9 marks}
Two trail elevation profiles give the vertical position (relative to a baseline) as a function of horizontal distance $x$ metres from the trailhead.

\begin{itemize}[leftmargin=2cm]
\item Trail A: $y = -3(x - 3)(x - 1), x \geq 0$ (elevation in metres)
\item Trail B: $y = (x - 1)(x - 2)(x - 4), x \geq 0$ (elevation in metres)
\end{itemize}

\textbf{a.} Where does Trail A meet the baseline (ground)? \hfill (1 mark)
\vspace{2cm}

\textbf{b.} Where does Trail B meet the baseline? \hfill (1 mark)
\vspace{2cm}

\textbf{c.} Find the horizontal distance $\bar{x}$ at which Trail A reaches its maximum elevation (give your answer in decimal form). \hfill (2 marks)
\vspace{3cm}

\textbf{d.} For which value(s) of $x$ are the two trails at the same elevation, and what is that common elevation (in metres)? \hfill (2 marks)
\vspace{4cm}

\textbf{e.} Hence or otherwise, sketch the graph of the polynomial: $y = x^3 - 7x^2 + 14x - 8$ in the interval $-1 \leq x \leq 5$. Clearly label all axis intercepts and endpoints. \hfill (3 marks)

\vspace{0.5cm}
\begin{center}
\includegraphics[width=0.7\textwidth]{new_images_CAT2/q15e_grid.png}
\end{center}
\vspace{0.5cm}
\end{question}

\newpage

% Question 16
\begin{question}{7 marks}
An investment fund models its net monthly return $R(x)$ (in thousands of dollars) by a monic cubic:

$$R(x) = x^3 + ax^2 + bx - 6$$

Where $x$ is the number of hundreds of clients in a month.

It is known that the fund breaks even when it has 200 clients, and that with 100 clients the fund produces a loss of \$8k.

\textbf{a.} Using this information, form two simultaneous equations in $a$ and $b$. \hfill (3 marks)
\vspace{4cm}

\textbf{b.} Solve your equations to find $a$ and $b$. \hfill (2 marks)
\vspace{4cm}

\textbf{c.} Write down the complete polynomial $R(x)$. \hfill (1 mark)
\vspace{2cm}

\textbf{d.} Hence, factor $R(x)$ fully and state all three linear factors. \hfill (1 mark)
\vspace{3cm}
\end{question}

\vspace{2em}
\begin{center}
\textbf{END OF TEST}
\end{center}

\end{document}
